\documentclass{article}

\usepackage[utf8]{inputenc}

\begin{document}
	\begin{center}
		\scshape
		\LARGE	
		Proyecto 1 - Altas Energ\'ias:\\
		
		\vspace{1cm}
		
		Respuesta de detectores MEDIPIX3 a diferentes espectros de 
		rayos X
		
		\vspace{0.5cm}
		\normalsize
		
		Juan Camilo Ramirez - 201312866 \\
		Juan Barbosa - 201325901
	\end{center}
	
	\begin{itemize}
		\item \textbf{Grupo investigaci\'on:} Altas Energ\'ias
		\item \textbf{Profesor responsable:} Carlos \'Avila
		\item \textbf{Estudiante o t\'ecnico de contacto:} Gerardo Roque
	\end{itemize}
	
	\section{Descripci\'on}
	El laboratorio de Altas Energías adquirió en el 2017 un nuevo tubo de rayos X con ánodo de Plata. También cuenta con un tubo de rayos X con ánodo de Tungsteno. Se desea hacer un estudio de la respuesta de estos detectores para los dos tubos de rayos X, hace un scan de energía en el rango de 10 keV hasta 100 keV y verificar el contraste de imagen que se obtiene para una misma muestra que es irradiada con los diferentes espectros de emisión.
	\section{Objetivos}
		\begin{enumerate}
			\item Probar el funcionamiento de la fuente de rayos X con ánodo de plata, recientemente adquirida por el laboratorio de altas energías.
			
			\item Hacer un scan de voltaje de las fuentes con ánodos de tungsteno y plata para verificar el voltaje y la intensidad de cada fuente que produce la imagen con el mejor contraste, dentro de los rangos de dosis permitidos para este tipo de imágenes.
			
			\item Hacer un estudio de cociente de señal  a ruido de las diferentes imágenes obtenidas para cada fuente de rayos X.
			
			\item Sacar conclusiones sobre el modo de operación y tipos de aplicaciones apropiados para ánodo de tungsteno y ánodo de plata.
		\end{enumerate}
\end{document}